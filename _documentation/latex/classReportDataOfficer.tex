\section{ReportDataOfficer Klassenreferenz}
\label{classReportDataOfficer}\index{ReportDataOfficer@{ReportDataOfficer}}
\subsection*{Öffentliche Methoden}
\begin{CompactItemize}
\item 
{\bf ReportDataOfficer} ()
\item 
{\bf loadGraphData} (\$controlId, \$sid)
\end{CompactItemize}


\subsection{Ausführliche Beschreibung}


Definiert in Zeile 9 der Datei class.ReportDataOfficer.php.

\subsection{Dokumentation der Elementfunktionen}
\index{ReportDataOfficer@{ReportDataOfficer}!ReportDataOfficer@{ReportDataOfficer}}
\index{ReportDataOfficer@{ReportDataOfficer}!ReportDataOfficer@{ReportDataOfficer}}
\subsubsection{\setlength{\rightskip}{0pt plus 5cm}ReportDataOfficer.ReportDataOfficer ()}\label{classReportDataOfficer_0cffec838a4f3c6c198349efa6b1477f}




Definiert in Zeile 11 der Datei class.ReportDataOfficer.php.\index{ReportDataOfficer@{ReportDataOfficer}!loadGraphData@{loadGraphData}}
\index{loadGraphData@{loadGraphData}!ReportDataOfficer@{ReportDataOfficer}}
\subsubsection{\setlength{\rightskip}{0pt plus 5cm}ReportDataOfficer.loadGraphData (\$ {\em controlId}, \$ {\em sid})}\label{classReportDataOfficer_3b60c363d8a6a385fc902e5651bbfdf3}


Load grapth data object

\begin{Desc}
\item[Parameter:]
\begin{description}
\item[{\em string}]\$controlId \item[{\em GraphData}]\$graphDataObj \end{description}
\end{Desc}


Definiert in Zeile 22 der Datei class.ReportDataOfficer.php.

Benutzt \$controlArr und \$controlId.

Die Dokumentation für diese Klasse wurde erzeugt aufgrund der Datei:\begin{CompactItemize}
\item 
{\bf class.ReportDataOfficer.php}\end{CompactItemize}
